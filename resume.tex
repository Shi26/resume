
%%%%%%%%%%%%%%%%%%%%%%%%%%%%%%%%%%%%%%%%%%%%%%
% resume - one page, one column
% XeLaTeX Template
% Version 1.1 (09.05.2017)
%
% Author: knyte (https://github.com/knyte)
%
% Repository: https://github.com/knyte/resume
%
% This template is primarily designed for
% undergraduate EE/CE/CS students applying for
% industry roles in software engineering
%
%%%%%%%%%%%%%%%%%%%%%%%%%%%%%%%%%%%%%%%%%%%%%%

\documentclass[]{resume-knyte}
\usepackage{hyperref}

\begin{document}

%%%%%%%%%
% HEADER
%%%%%%%%%

% \makeheader{title/name}
\makeheader{Cheng Zhang}

%%%%%%%%%%%%
% SUBHEADER
%%%%%%%%%%%%

% argument to environment should match number of sites
\begin{subheader}{1}
% leave third argument empty to specify no hyperlink
    \site{GitHub}{Chantisnake}{https://github.com/chantisnake}
    \site{Email}{zcsx.ok at outlook dot com}{mailto://zcsx.ok@outlook.com}
    \site{Phone}{(+1)508-818-8290}{}\\

\end{subheader}

%%%%%%%%%%%%
% EDUCATION
%%%%%%%%%%%%
\begin{topic}{Education}

    % \entry{institution}{dates}{subtitle}{description}

    \entry{Wheaton College}{Sep. 2014 - Present}
    {Bachelor of Art in Mathematics}
    {Major in Mathematics, Minor in Computer Science and Economics\\
    Will complete 42 credit hours before graduate. Current Major GPA: 3.9, General GPA: 3.84\\
    Honors: Dean's List (for all the 4 semesters), Wheaton Fellow, Faculty Student Grant.}


    \entry{London School Of Economics}{Sep. 2016 - Jun. 2017}
    {Study Aboard}
    {Took EC221 Principles of Econometrics, EC202 Microeconomic Principles II,
    MA300 Game Theory, MA211 Algebra and Number Theory, MA315 Algebra and its Applications
    }

    \\ % hard coded to adjust spacing
\end{topic}




%%%%%%%%%%%%%
% EXPERIENCE
%%%%%%%%%%%%%
\begin{topic}{Experience}

    % \entry{role}{dates}{team/company}{description}

    \entry{Software Leader}{Jun. 2015 - Dec. 2017}
    {Lexomics Research Team}
    {Lead the development of \href{https://github.com/WheatonCS/Lexos}{Lexos}, a web app for text analysis workflow.
    Help the team to adopt modern software development paradigm and Workflow.
    Designed a new architecture and a new python style guide for the project.}

    \entry{Mathematics Research}{Jun. 2016 - Jul. 2016}
    {Wheaton College Mathematics Department}
    {Work with \href{https://wheatoncollege.edu/academics/faculty-directory/rochelle-shelly-leibowitz/}{Prof. Leibowitz}
    on the topic of \href{https://chantisnake.github.io/2016/05/22/explain-the-divisibility-graph/}{Divisibility Graph}.}

    \entry{Mathematics Research}{Sep. 2017 - Jun. 2018}
    {Honor Thesis}
    {Research on graph tournament, inspired by \href{https://www.maa.org/sites/default/files/pdf/upload_library/22/Allendoerfer/1981/0025570x.di021114.02p00982.pdf}{King Chicken Theorem}.
    Our Research focus on the kings behavior in improper tournaments (tournaments with ties).}


    \entry{Open-source Contributor}{Sep. 2014 - Present}
    {Contributor to Many Open Source Project}
    {Involved in the development of \href{https://github.com/janikvonrotz/awesome-powershell}{Awesome-Powershell}, \href{https://github.com/pcgeek86/PSGitHub}{PSGitHub}.
    Contribute Ideas to \href{https://github.com/python/typing}{python/typing}, \href{https://github.com/python/mypy}{mypy} and \href{https://github.com/siegebell/vscoq}{VSCoq}.
    Help to provide Minor improvement to \href{https://github.com/FStarLang/FStar}{FStar} and many other Projects.}

\\ % add spacing
\end{topic}



%%%%%%%%%
% SKILLS
%%%%%%%%%
\begin{topic}{Skills}

    % \skillset{category}{members}

    \skillset{Mathematics Fields}
    {Experienced in Graph Theory.\\
    Familiar with:\\
    - Topology, Complex Analysis, Real Analysis and Algebra.\\
    - Combinatorics, Game Theory, Probability and Mathematical Statistics.}

    \skillset{Programming Languages}
    {Experienced in Python and Powershell.\\
    Familiar with Haskell, F\#, Coq, TypeScript, JavaScript and \LaTeX{}.\\
    Knows the basic about idris, Scala, Elm.}


    \\ % hardcoded due to entry spacing
\end{topic}



%%%%%%%%%%%
% PROJECTS
%%%%%%%%%%%
\begin{topic}{Projects}

    % \entry{name}{dates}{tools used}{description}

    \entry{Markdown for Academia}{Mar. 2017 - Apr. 2017}
    {An easy to use markup language designed for academic writing}
    {Semitics based on pandoc markdown, added many more feature to support academic writing.\\
    Can be compiled easily to HTML, PDF, \LaTeX{} and many other formats.}

\end{topic}

\end{document}

